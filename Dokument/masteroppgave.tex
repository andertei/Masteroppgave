\documentclass[10pt,a4paper,twoside]{article} 
\usepackage[latin1]{inputenc}
\usepackage[english]{babel}
\usepackage{amsmath}
\usepackage{amsfonts}
\usepackage{amssymb}
\usepackage{makeidx}
\usepackage{graphicx}
\usepackage{hyperref}
\usepackage[left=2cm,right=2cm,top=2cm,bottom=2cm]{geometry}
\usepackage{float}
\usepackage{multirow}
\usepackage{verbatim} %for å kommentere ut ting
\usepackage[nottoc,numbib]{tocbibind}
\usepackage[parfill]{parskip} %for avsnitt

\raggedbottom

\usepackage{makeidx}
\makeindex
%symbolliste slutt

\author{Anders Dall'Osso Teigset}
\title{MEDIUM VOLTAGE LOAD BREAK SWITCH WITH AIR AS INTERRUPTING MEDIUM}
\date{December, 2013}


\begin{document}
    \begin{titlepage}
    \begin{center}
    \ \\
    \ \\
    \ \\
    \ \\
    \ \\
    \ \\
    Anders Dall'Osso Teigset \\
    \ \\
    \ \\
    \ \\
    \ \\{\large \bfseries
    MEDIUM VOLTAGE LOAD BREAK SWITCH WITH AIR AS INTERRUPTING MEDIUM\\
    }
    \ \\
    \ \\
    \ \\
    \ \\
    \ \\
    {\large
    Specialisation project\\
    }
    \ \\
    {December, 2013 \\}
    \ \\
    \ \\
    \ \\
    \ \\
    \ \\
    \ \\
    \ \\
    \ \\
    \ \\
    \ \\
    \ \\
    \ \\
    \ \\
    \ \\
    \ \\
    \ \\
    \ \\
    \ \\
    \ \\
    \ \\
    \ \\
    \ \\
    \ \\
    \ \\
    \ \\
    \ \\
    \ \\
    \ \\
    \ \\
   	{\large
   Norwegian University of Science and Technology\\
   Department of Electric Power Engineering\\
    }
   	\ \\
    \ \\
    \ \\
    \ \\
    \end{center}
    \end{titlepage}

%\maketitle
%SummaryNyttige pdf-filer/SF6conduct.pdf
\thispagestyle{empty}
\cleardoublepage
\section*{Acknowledgements}
\setcounter{page}{1}
\pagenumbering{roman}

\cleardoublepage
\section*{Summary}

\cleardoublepage
\setcounter{page}{1}
\pagenumbering{arabic}
\tableofcontents
\cleardoublepage

\section{Introduction}

\cleardoublepage

\section{Theory}
\subsection{Typical switchgear design and interruption sequence}


\subsubsection{Switchgear design and operation} 


\subsubsection{The puffer principle}

   
\subsubsection{Heat transportation in an arc} 


\subsubsection{The difference between air and SF$_6$ as interrupting medium} 



\subsubsection*{Electrical conductivity}


\subsubsection*{Thermal conductivity}


\subsubsection*{Dielectric properties} 


\subsubsection*{Chemical properties}


\subsection{Environmental impacts of SF$_6$ from electrical power industries} 


\cleardoublepage

\section{Method}

\subsection{Test circuit}

Figure \ref{fig:testSwitchRiggEq} illustrates the physical appearance of the test switch. The numbered parts are: 1. Compressed air reservoir (connected to the high voltage supply circuit), 2. Tulip contact, 3. Nozzle, 4. Pin contact, 5. Connection to load circuit, 6. Spring drive mechanism, 7. Electromagnet release mechanism, and 8. Position transducer.

\begin{figure} [H]
\centering
\includegraphics[scale=0.5]{Bilder/Method/switchTest.png}
\caption{The physical appearance of the test switch \cite{bib:AFIMVLBA}.} \label{fig:testSwitchRiggEq}
\end{figure}

Figure \ref{fig:testCurcuit} displays the laboratory test circuit used for the interruption tests. The circuit is designed to supply a 50 Hz / 13.8 kV current. It is possible to shape the TRV by tuning the parameters: L$_1$, L$_s$, R$_1$, R$_d$, and C. The systems' short circuit parameters are R$_{sc}$ and L$_{sc}$. The TRV generated during interruption is set to simulate the standard for a 24 kV / 630 A class from the International Electrotechnical Commission (IEC), which corresponds to:

\begin{itemize}
\item The initial part of the TRV has a rate of rise in recovery voltage (RRRV) of 71 - 73 V / $\mu$s.
	\begin{itemize}
		\item Voltage difference is measured over the first 20 $\mu$s after CZ.
	\end{itemize}
\item The first voltage peak is between 7.0 and 7.4 kV, with a rise time of approximately 96 $\mu$s.
\end{itemize}

\begin{figure} [H]
\centering
\includegraphics[scale=0.35]{Bilder/Method/circuit.png}
\caption{Circuit used for the interruption test \cite{bib:AFIMVLBA}.} \label{fig:testCurcuit}
\end{figure}

In table \ref{tab:testParameters}, the value of the different test circuit parameters and the corresponding current can be observed. The test is done at currents with an RMS value of 400 A and 630 A. In the entire test the TRV is kept constant up to and including the first voltage peak. In the case of a failed interruption, thermal re-ignition occurs within a few microseconds after CZ.

\begin{table}[H]
\center
\caption{Circuit Parameters and Resulting Current \cite{bib:AFIMVLBA}. }
\begin{tabular}{|c|c|c|c|c|c|}
\hline 
L$_s$ [mH] & L$_1$ [mH] & R$_1$ [$\Omega$] & C [nF] & R$_{d}$ [$\Omega$] & I [A] \\ 
\hline 
14.5 & 138.4 & 35.5 & 74 & 248 & 400 \\ 
\hline 
6.9 & 86.2 & 22.1 & 102 & 198 & 630 \\ 
\hline 
\end{tabular} 
\label{tab:testParameters}
\end{table}

A resistive transducer is measuring the contact position while a Hall Effect current transducer is measuring the current through the test switch. The voltage between the contacts is measured with a parallel resistive / capacitive voltage divider. All measurements are transmitted through optical fibres to a 12 bit resolution transient recorder with a sampling frequency of 2.5 MHz. The pressure in the tank is only measured before each test with an accuracy of 0.01 bar. The test switch is displayed in figure \ref{fig:testSwitchRiggEq}.

\subsection{The switch and contact geometry}



\newpage
\subsection{Procedure}


\cleardoublepage

\section{Results}
\subsection{Interruption tests} 


\newpage
\subsection{Arcing voltage}


\newpage
\subsection{Durability of the arcing contacts} \label{sec:durability}



\cleardoublepage

\section{Discussion}
\subsection{The probability of interruption} 


\subsection{Arcing voltage considerations}
 

\subsection{Durability of the arcing contacts} \label{fig:durability}


\newpage
\subsection{Suggestion for further work}
\subsubsection{A nozzle that minimises arc impact on air flow}


\subsubsection{Cone-shaped nozzle}


\cleardoublepage

\section{Conclusion}


\cleardoublepage
\begin{thebibliography}{10}
\bibitem{bib:SF6PI} L.G. Christophorou, J. K. Olthoff, and R.J. Van Brunt, "Sulfur hexafluoride and the electric power industry", \textit{IEEE Electrical Insulation Magazine, vol. 13, No. 5, pp. 20-24}, Oct. 1997.

\bibitem{bib:comSub} amesimpex.com, \url{http://www.amesimpex.com/images/unitised_sub_002.jpg}, 26.9.2013

\bibitem{bib:HVEbreak} M. Runde, "Current interruption in power grids", Trondheim: Norwegian University of Science and Technology, 2013

\bibitem{bib:CBAC} W. Rieder, "Circuit breakers, physical and engineering problems, III-Arc-medium considerations", \textit{IEEE spectrum, pp. 80-84}, Sept. 1970.

\bibitem{bib:IPSF6AQM} W. Hertz, H. Motschmann and H. Wittel, "Investigations of the properties of SF$_6$ as an arc quenching medium", \textit{Proceedings of The IEEE, vol. 59, NO. 4, pp. 485-492}, April 1971.

\bibitem{bib:TDCIGBB} W. Hermann, "Theoretical description of the current interruption in HV gas blast breakers", \textit{IEEE Transactions on Power Apparatus and System, vol. PAS-96, NO. 5, pp. 1546-1555}, Sept./ Oct. 1977.

\bibitem{bib:THFD} R. W. Johnson, "The handbook of fluid dynamics", Heidelberg: Springer-Verlag GmbH \& Co. KG, 1998.

\bibitem{bib:TET4160HVIM} E. Ildstad, "High voltage insulation materials", Trondheim: Norwegian University of Science and Technology, 2012, August 2012.

\bibitem{bib:KlimaKur2020} "KLIMAKUR2020", Oslo: Klima- og forurensningsdirektoratet, 2010

\bibitem{bib:consSF6} esrl.noaa.gov, \url{http://www.esrl.noaa.gov/gmd/webdata/iadv/ccgg/graphs/pdfs/ccgg.MLO.sf6.1.none.discrete.all.pdf}, 17.10.2013

\bibitem{bib:regSF6Miljo} regjeringen.no, \url{http://www.regjeringen.no/nb/dep/md/dok/regpubl/stmeld/2011-2012/meld-st-21-2011-2012/5/5.html?id=682932}, 21.10.2013

\bibitem{bib:StatSF6} K. L. Hansen, "Emissions from consumption of HFCs, PFCs and SF$_6$ in Norway", \textit{Statistics Norway/Department of Economic Statistics/Environmental Statistics}, 2007.

\bibitem{bib:AFIMVLBA} N. S. Aanensen, E. Jonsson, and M. Runde "Air flow investigation for a medium voltage load break switch", to be published.

\bibitem{bib:CIAMVLBS} E. Jonsson, N. S. Aanensen and M. Runde, "Current interruption in air for a medium voltage load break switch", \textit{IEEE Trans. Power Delivery}, in press.

\end{thebibliography}

\cleardoublepage
\appendix
\vspace*{\fill}
\begingroup
\begin{center}
\huge Appendix
\end{center}
\endgroup
\vspace*{\fill}
\cleardoublepage
\section{Appendix: Test Results} \label{app:rawData}
\setcounter{figure}{0}
\makeatletter 
\renewcommand{\thefigure}{A.\@arabic\c@figure}
\makeatother

\setcounter{table}{0}
\makeatletter 
\renewcommand{\thetable}{A.\@arabic\c@table}
\makeatother

\subsection{400 A geometry \textit{a} and \textit{b}} \label{app:testResults400A} 

\subsection{630 A geometry \textit{a} and \textit{b}} \label{app:testResults630A}

\newpage



\cleardoublepage
\section{Appendix: Previous relevant experiment} \label{app:PrevReleEx}
\makeatletter 
\renewcommand{\thefigure}{B.\@arabic\c@figure}
\makeatother

\makeatletter 
\renewcommand{\thetable}{B.\@arabic\c@table}
\makeatother


\end{document}
